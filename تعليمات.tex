    الحدود (Borders)
    border-b: يضيف إطارًا في الأسفل.
    border-t: يضيف إطارًا في الأعلى.
    border-r: يضيف إطارًا في اليمين.
    border-l: يضيف إطارًا في اليسار.
    border-gray-300: يحدد لون الإطار باللون الرمادي الفاتح.
    الهوامش الداخلية (Padding)
    p-: هوامش داخلية (padding).
    pb-2: يحدد حشوة سفلية (padding bottom) بـ 2 وحدة.
    pt-: هوامش علوية (padding top).
    p-4: يحدد حشوة بمقدار 4 وحدات من جميع الجهات.
    p-2: يحدد حشوة بمقدار 2 وحدة من جميع الجهات.
    النصوص (Text)
    text-xl: يحدد حجم الخط ليكون كبيرًا.
    text-lg: حجم الخط متوسط.
    text-base: حجم الخط أساسي.
    text-sm: حجم الخط صغير.
    text-xs: حجم الخط صغير جداً.
    font-bold: يجعل النص سميكًا.
    font-medium: نص متوسط.
    font-normal: نص عادي.
    font-light: نص خفيف.
    محاذاة النص (Text Alignment)
    text-center: يوسّط النص.
    text-right: محاذاة لليمين.
    text-left: محاذاة لليسار.
    text-blue-700: لون الخط أزرق.
    عرض الطباعة (Print Display)
    hidden print:block: يخفي العنصر في العرض العادي ويظهره فقط عند الطباعة.
    print:hidden: يخفي العنصر عند الطباعة.
    الهوامش الخارجية (Margin)
    m-: إضافة هوامش خارجية (margin).
    mb-4: يحدد هامش سفلي بـ 4 وحدات (margin-bottom).
    mt-4: يحدد هامش علوي بـ 4 وحدات (margin-top).
    mb-2: يحدد هامش خارجي سفلي (margin bottom) بـ 2 وحدة.
    margin: 20px 0;: يحدد الهوامش الخارجية بـ 20 بكسل في الأعلى والأسفل و0 بكسل على الجانبين.
    هام
    my-2:  تُطبق هامشًا خارجيًا في الأعلى والأسفل (Margin-Y)
    الزوايا المدورة (Rounded Corners)
    rounded-lg: يضيف زوايا مدورة بشكل كبير.
    rounded-md: يضيف زوايا مدورة بشكل متوسط.
    rounded-full: يضيف زوايا مدورة بشكل كامل.
    خلفيات (Background)
    bg-: لون الخلفية.
    bg-gray-50: لون خلفية رمادي.
    الفليكس (Flex)
    flex-1: يتيح للعنصر أن يأخذ المساحة المتاحة في الفليكس.
    الشبكة (Grid)
    grid: يحدد العنصر كشبكة.
    grid gap-4: يحدد المسافة جانبية او عمودية بين العناصر في الشبكة بمقدار 4 وحدات. 
    الفجوات (Gap)
    gap-: يحدد المسافة بين العناصر.
    gap-1: لتحكم بالمسافات الأفقية بين العناصر.
    محاذاة العناصر (Justification)
    justify-center: يوسّط العناصر داخل الحاوية.
    تعطيل العنصر (Disabled)
    disabled:opacity-: يحدد مستوى الشفافية عند تعطيل العنصر.
    النصوص المتقطعة (Whitespace)
    whitespace-nowrap: يمنع النص من النزول إلى سطر جديد.
    ارتفاع السطور (Line Height)
    line-height: المسافة بين السطور.
    إضافات أخرى
    space-y-0: لإزالة المسافات الرأسية بين العناصر.
    gap-1: لتحكم بالمسافات الأفقية بين العناصر.
    text-xs: لتصغير حجم الخط.
    h-6: لتقليل ارتفاع العناصر.
    w-32: لتحكم بعرض العناصر.
    m-0 و print:p-0: لإزالة الهوامش والمسافات الداخلية.
    التركيز والتأثيرات (Focus and Transitions)
    focus:outline-none: إزالة الحدود عند التركيز.
    focus:ring-2: إضافة حلقة عند التركيز بسماكة 2.
    focus:ring-blue-500: تحديد لون الحلقة باللون الأزرق.
    focus:border-blue-500: تحديد لون الحدود عند التركيز باللون الأزرق.
    transition-colors: إضافة تأثيرات انتقالية على الألوان.

    خصائص التحكم في العرض والطول
    عرض العناصر (Width)

    w-: لتحديد عرض العنصر. مثل:
    w-full: لملء العرض الكامل.
    w-1/2: لعرض نصف العنصر.
    w-64: لتحديد عرض ثابت بـ 16 وحدة (بكسل).
    ارتفاع العناصر (Height)

    h-: لتحديد ارتفاع العنصر. مثل:
    h-full: لملء الارتفاع الكامل.
    h-1/2: لارتفاع نصف العنصر.
    h-32: لتحديد ارتفاع ثابت بـ 8 وحدات.
    min-width و min-height

    min-w-: لتحديد الحد الأدنى للعرض.
    min-h-: لتحديد الحد الأدنى للارتفاع.
    max-width و max-height

    max-w-: لتحديد الحد الأقصى للعرض.
    max-h-: لتحديد الحد الأقصى للارتفاع.
    Flexbox

    flex: لتحديد العنصر كحاوية فليكس.
    flex-grow: لتحديد كيفية توزيع المساحة المتاحة بين العناصر.
    flex-shrink: لتحديد كيفية تقليص العناصر عندما يكون هناك نقص في المساحة.
    Grid

    grid-cols-: لتحديد عدد الأعمدة في شبكة.
    grid-rows-: لتحديد عدد الصفوف في شبكة.
    Aspect Ratio

    aspect-ratio-: لتحديد نسبة العرض إلى الارتفاع، مثل aspect-ratio-16/9.
    فوائد استخدام هذه الخصائص
    استجابة التصميم: يمكن استخدامها لجعل العناصر تتكيف مع أحجام الشاشات المختلفة.
    تحسين تجربة المستخدم: يساعد في تنظيم العناصر بشكل جيد ويسهل القراءة والتفاعل.
    تحكم أكبر: يتيح لك ضبط العناصر بدقة وفقًا لمتطلبات التصميم.
    استخدام هذه الخصائص يساهم في تحسين تصميم الصفحة وجعلها أكثر تفاعلية وجاذبية.


    p-4: يضيف حشوة داخلية (padding) بمقدار 1rem (16px) من جميع الاتجاهات.
    mb-4: يضيف هامش سفلي (margin-bottom) بمقدار 1rem (16px).
    shadow-none: يزيل الظل عند الطباعة.
    p-2: يغير الحشوة إلى 0.5rem (8px) عند الطباعة.
    border border-gray-200: يضيف إطارًا بلون رمادي فاتح حول العنصر.
    print-container: قد يكون هذا كلاس مخصص لتنسيق خاص عند الطباعة.
    <div className="flex items-center justify-between mb-4">:

    flex: يجعل العنصر حاوية مرنة (flex container).
    items-center: يقوم بمحاذاة العناصر العمودية في المنتصف.
    justify-between: يوزع العناصر بحيث يكون هناك مساحة متساوية بينهما.
    mb-4: يضيف هامش سفلي (margin-bottom) بمقدار 1rem (16px) أيضًا.
    <h2 className="text-lg font-semibold flex items-center text-gray-800">:

    text-lg: يحدد حجم النص بأنه كبير.
    font-semibold: يجعل الخط سميكًا قليلاً.
    flex items-center: يجعل العنوان عنصرًا مرنًا مع العناصر المحاذية له.
    text-gray-800: يحدد لون النص بلون رمادي داكن.
    <DollarSign className="ml-2" size={20} />:

    تمثل أيقونة الدولار، مع إضافة هامش يسار (margin-left) بمقدار 0.5rem (8px)











    إزالة المسافة بين مربع الموردين والتوقيعات margin-top: 30px;

    whitespace-nowrap → يمنع النص من النزول لسطر جديد في السطر

    border-2: يحدد سمك الإطار بـ 2 بكسل.
    border-gray-400: يحدد لون الإطار بلون رمادي متوسط.
    rounded-lg: يضيف زوايا مدورة بشكل كبير.
    pt-2: يحدد هوامش (مسافة) علويّة (padding top) بـ 2 وحدة.
    print:border-2: يحدد سمك الإطار بـ 2 بكسل عند الطباعة.
    print:border-black: يغير لون الإطار إلى الأسود عند الطباعة.
    print:rounded-none: يزيل الزوايا المدورة عند الطباعة.


    margin: 20px 0;:
    يحدد الهوامش الخارجية. هنا، يتم تعيين هامش قدره 20 بكسل في الأعلى والأسفل (العمودي) و0 بكسل على الجانبين (الأفقي).

    padding: 15px;:
    يحدد الهوامش الداخلية. يتم تعيين حشو قدره 15 بكسل من جميع الجوانب (الأعلى، الأسفل، اليمين، واليسار).

    border: 1px solid #ccc;:
    يحدد إطار العنصر. هنا، يتم تعيين إطار بسماكة 1 بكسل، وبشكل صلب (solid) ولونه رمادي فاتح (#ccc).
